\documentclass[20pt]{article}
\usepackage{mathtools}
\usepackage{amsmath}
\usepackage{siunitx}
\usepackage{commath}
\usepackage{graphicx}
\usepackage{float}
\usepackage{fancyhdr}
\pagestyle{fancy}
\fancyhf{}
\fancyhead[LE,RO]{Notes: C4 from models and data}
\fancyhead[RE,LO]{Winter 2016}
\begin{document}
From Kam's work, we know that the autocorrelator is 
\begin{equation}
	C(q,\psi) = <I(\vec{q_1})I(\vec{q_2})>-<I(\vec{q_1})><I(\vec{q_2})>
\end{equation}
where $\vec{q_1}$ and $\vec{q_2}$ have the same magnitude $q$. To compute this from MD simulation we have:
\begin{equation} \label{eq:est_corr}
 <I(\vec{q_1})I(\vec{q_2})>  = \frac{ 4 \abs{f(q)}^4 }{N_{tthd}} \sum^{N}_{i,j,k,l}  (1+\cos{\vec{q_1} \cdot \vec r_{ij}} ) (1+\cos{\vec{q_2} \cdot \vec r_{kl}} )
\end{equation}
and
\begin{equation} \label{eq:est_corr}
 <I(\vec q)> = \frac{ 2 \abs{f(q)}^2}{N_{pair}} \sum^{N}_{i,j}  (1+\cos{\vec q \cdot \vec r_{ij}} )
\end{equation}
where $i$, $j$, $k$, $l$ are indices representing the vertices of the nearest-neighbor tetrahedrons. I use a two-vector representation of a tetrahedron, i.e each is denoted by $\{\vec r_{ij},\vec r_{lk}\}$, $\vec r_{ij} = \vec r_i- \vec r_j$ etc. That means for each set of neighbors, there are three different two-vector presentations. I will deal with how this will impact the calculation of $C(q,\psi)$ later. The averages are over all the tetrahedrons/pairs. 

I used simulation ran laster quarter (Fall 2015) and the results are plotted in figure~\ref{fig:autocorr}.
\begin{figure}[!h] 
  \centering
    \includegraphics[width=1.0\textwidth]{../output/autocorr_cos_psi_run7.jpg}
     \caption{Autocorrelator $ C(q,\cos \psi)$ computed from simulated water} \label{fig:autocorr}
\end{figure}
I had a similar plot in the Fall 2015 notes that is just the $ <I(\vec{q_1})I(\vec{q_2})>$ term. Subtracting the $<I(\vec{q_1})><I(\vec{q_2})>$ "normalizes" the autocorrelator.

\begin{figure}[!h] 
  \centering
    \includegraphics[width=1.0\textwidth]{../output/rdf_mdtraj_run8.png}
     \caption{Radial distribution function of simulated water.} \label{fig:rdf_mdtraj}
\end{figure}

Examining the radial distribution function in figure~\ref{fig:rdf_mdtraj} tells me the upper bound of $q$ I might expect good sampling for tetrahedron configurations. The ref drops to zero at around 0.25 nm which translated into about 2.5 $\AA^{-1}$. I also know from finding nearest-neighbor tetrahedrons in MD simulations that $\abs{\vec r_{ij}}$'s are no more than 0.5 nm. So the lower bound of $q$ should be around 1.3 $\AA^{-1}$. So the range of $q$ I have looked at are between 1.3 to 2.5 $\AA^{-1}$.

\section{two-vector representation of tetrahedrons}
One question to ask here is: does the two-vector representation $\{\vec r_{ij},\vec r_{lk}\}$ affect the result of $C(q, \psi)$? If I have vertices labeled, 1, 2, 3, and 4, I can have three grouping if I am to divide them into two pairs, namely $\{(1, 2), (3, 4)\}$, $\{(1, 3), (2, 4)\}$, and $\{(1, 4), (2, 3)\}$. So every tetrahedron has three unique ways to scatter two photons (strictly speaking I think it have $3 \times 2 = 6$ way for two photons). The question is will $C(q, \psi)$ be dependent on which grouping I choose.

My understanding is it does not but the grouping allows me to get more configurations to average over. Figure~\ref{fig:tthdsets_compare} show that the curves computed using different groups converges to their average with one sigma.
\begin{figure}[!h] 
  \centering
    \includegraphics[width=1.0\textwidth]{../output/run3_compare_tthdsets_q135.jpg}
     \caption{Comparing $C(q \psi)$ computed from different two-vector representations of the tetrahedron, $q = 1.35 \AA^{-1}$.} \label{fig:tthdsets_compare}
\end{figure}

\section{Projection of C4 into Legendre polynomials}
How am I going to compare the simulation to the CXS data? Ultimately, I am hoping that the CXS is good enough for me to distinguish between models for water out there (TIP3, TIP4, SPC, SPC/E, etc.). 

Again, Kam's paper tells me that I can expand $C(q, \psi)$ in terms of the Legendre polynomials $P_l(\cos \psi)$:
\begin{equation}
C(q, \psi) = \sum_l B_l(q) P_l(\cos \psi)
\end{equation}

I need error bars on the coefficient plot. I need to estimate how much simulation do I need to do to be able to distinguish $C$ amongst the different force fields/ models for water.
\end{document}